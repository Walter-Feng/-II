\documentclass[a4paper,12pt]{article}
%%%%%%%%%%%%%%%%%%%%%%%%%%%%%%%%%%%%%%%%%%%%%%%%%%%%%%%%

%Packages

%%%%%%%%%%%%%%%%%%%%%%%%%%%%%%%%%%%%%%%%%%%%%%%%%%%%%%%
\usepackage[utf8]{inputenc}

\usepackage{graphicx}
\usepackage{subfigure}
\usepackage{multirow}
\usepackage{array}
\graphicspath{{figures/}} % Location of the graphics files
\usepackage{booktabs} % Top and bottom rules for table不【
\usepackage[font=small,labelfont=bf]{caption} % Required for specifying captions to tables and figures
\usepackage{amsfonts, amsmath, amsthm, amssymb} % For math fonts, symbols and environments
\usepackage{wrapfig} % Allows wrapping text around tables and figures
\usepackage[centering]{geometry}

\usepackage{ctex}
\usepackage{textcomp}
\usepackage[final]{pdfpages}
\geometry{top=25.4mm, bottom=25.4mm, left=31.8mm, right=31.8mm}



\usepackage{titlesec}
\titleformat{\section}{\sihao\bfseries}{\thesection.}{1em}{}
\titleformat{\subsection}{\xiaosi\bfseries}{\thesubsection}{1em}{}
\titleformat{\subsubsection}{\xiaosi}{\thesubsubsection}{1em}{}

\usepackage{chemfig}

\usepackage{fontspec} % 定制字体

\setmainfont{Times New Roman}

\usepackage{xunicode,xltxtra}
\usepackage{setspace}
\newcommand{\chuhao}{\fontsize{42pt}{44.9pt}\selectfont}    % 初号, 1.5倍行距
\newcommand{\xiaochu}{\fontsize{30pt}{40pt}\selectfont}    % 小初, 1.5倍行距
\newcommand{\yihao}{\fontsize{26pt}{36pt}\selectfont}    % 一号, 1.4倍行距
\newcommand{\erhao}{\fontsize{22pt}{28pt}\selectfont}    % 二号, 1.25倍行距
\newcommand{\xiaoer}{\fontsize{18pt}{18pt}\selectfont}    % 小二, 单倍行距
\newcommand{\sanhao}{\fontsize{16pt}{24pt}\selectfont}    % 三号, 1.5倍行距
\newcommand{\xiaosan}{\fontsize{15pt}{22pt}\selectfont}    % 小三, 1.5倍行距
\newcommand{\sihao}{\fontsize{14pt}{21pt}\selectfont}  
\newcommand{\sihaox}{\fontsize{14pt}{28pt}\selectfont}   % 四号, 1.5倍行距
\newcommand{\banxiaosi}{\fontsize{13pt}{19.5pt}\selectfont}    % 半小四, 1.5倍行距
\newcommand{\xiaosi}{\fontsize{12pt}{18pt}\selectfont}    % 小四, 1.5倍行距
\newcommand{\dawuhao}{\fontsize{11pt}{11pt}\selectfont}    % 大五号, 单倍行距
\newcommand{\wuhao}{\fontsize{10.5pt}{15.75pt}\selectfont}    % 五号, 单倍行距
\newcommand{\xiaowu}{\fontsize{9pt}{9pt}\selectfont}    % 五号, 单倍行距

\usepackage{listings}
\usepackage{xcolor} % 定制颜色
\definecolor{mygreen}{rgb}{0,0.6,0}
\definecolor{mygray}{rgb}{0.5,0.5,0.5}
\definecolor{mymauve}{rgb}{0.58,0,0.82}
\lstset{
backgroundcolor=\color{white},      % choose the background color
basicstyle=\footnotesize\ttfamily,  % size of fonts used for the code
columns=fullflexible,
tabsize=4,
breaklines=true,               % automatic line breaking only at whitespace
captionpos=b,                  % sets the caption-position to bottom
commentstyle=\color{mygreen},  % comment style
escapeinside={\%*}{*)},        % if you want to add LaTeX within your code
keywordstyle=\color{blue},     % keyword style
stringstyle=\color{mymauve}\ttfamily,  % string literal style
frame=single,
rulesepcolor=\color{red!20!green!20!blue!20},
% identifierstyle=\color{red},
language=Mathematica,
}

\usepackage[version=4]{mhchem}

%%%%%%%%%%%%%%%%%%%%%%%%%%%%%%%%%%%%%%%%%%%%%%%%%%%%%%%%

%Commands
\def\celsius{\ensuremath{^\circ\hspace{-0.09em}\mathrm{C}}}

%This means in the text, you can just type "\compoundA" or "\compoundB" between words
% to print out K5CoW12O40 or K6CoW12O40. Like..."We obtained quite a lot of \compoundA and %blabla.

%%%%%%%%%%%%%%%%%%%%%%%%%%%%%%%%%%%%%%%%%%%%%%%%%%%%%%%%%%%%%

\renewcommand\thesection{\arabic{section}} 

\title{甲酸氧化反应动力学探究}


\begin{document}
\makeatletter
\newcommand\dlmu[2][4cm]{\hskip1pt\underline{\hb@xt@ #1{\hss#2\hss}}\hskip3pt}
\makeatother

{\sihaox\fangsong
\begin{titlepage}
\vspace{2cm}
\begin{center}
\includegraphics[width=0.5\textwidth]{preface/Zhejianguniversitytitle.png}\\[1cm]

\includegraphics[width=0.2\textwidth]{preface/timg-2.jpeg}

\sanhao{\textbf{本~科~实~验~报~告}}\\[2cm]
\end{center}




\begin{minipage}{0.2\textwidth}
\begin{flushright}
姓名:

学院:

系:

专业:

学号:

指导教师:
\end{flushright}
\end{minipage}
\begin{minipage}{0.7\textwidth}
\begin{center}
\dlmu[0.8\linewidth]{黄宗玮,李一刘,秦宇轩,吴冠履,李睿,王永尧}

\dlmu[0.8\linewidth]{竺可桢学院}

\dlmu[0.8\linewidth]{化学系}

\dlmu[0.8\linewidth]{求是科学班(化学)}

\dlmu[0.8\linewidth]{}

\dlmu[0.8\linewidth]{王永尧}
\end{center}
\end{minipage}\\[2cm]

\begin{center}
\number\year 年 \number\month 月 \number\day 日
\end{center}
\end{titlepage}
}
\newpage

%%%%%%%%%%%%%%%%%%%%%%%%%%%%%%%%%%%%%%%%%%%%%%%%%%%%%%%%%%%%%
%Contents Table
%%%%%%%%%%%%%%%%%%%%%%%%%%%%%%%%%%%%%%%%%%%%%%%%%%%%%%%%%%%%%
\renewcommand{\contentsname}{目录}
\tableofcontents
\newpage

%%%%%%%%%%%%%%%%%%%%%%%%%%%%%%%%%%%%%%%%%%%%%%%%%%%%%%%%%%%%%


\bibliographystyle{unsrt}

%%%%%%%%%%%%%%%%%%%%%%%%%%%%%%%%%%%%%%%%%%%%%%%%%%%%%%%%%%%%%

%Title

%%%%%%%%%%%%%%%%%%%%%%%%%%%%%%%%%%%%%%%%%%%%%%%%%%%%%%%%%%%%%
\begin{center}
\xiaosan\textbf{甲酸氧化反应动力学探究}
\end{center}

%%%%%%%%%%%%%%%%%%%%%%%%%%%%%%%%%%%%%%%%%%%%%%%%%%%%%%%%%%%%%

%Author

%Please Type your name with your student number

\begin{center}
\xiaosi{黄宗玮,李一刘,秦宇轩,吴冠履,李睿,王永尧}
\end{center}
%%%%%%%%%%%%%%%%%%%%%%%%%%%%%%%%%%%%%%%%%%%%%%%%%%%%%%%%%%%%%

%%%%%%%%%%%%%%%%%%%%%%%%%%%%%%%%%%%%%%%%%%%%%%%%%%%%%%%%%%%%%


%Abstract & KeyWord

%%%%%%%%%%%%%%%%%%%%%%%%%%%%%%%%%%%%%%%%%%%%%%%%%%%%%%%%%%%%%
\wuhao{
\textbf{摘要:} 

\textbf{关键词:}
}
%%%%%%%%%%%%%%%%%%%%%%%%%%%%%%%%%%%%%%%%%%%%%%%%%%%%%%%%%%%%%


%%%%%%%%%%%%%%%%%%%%%%%%%%%%%%%%%%%%%%%%%%%%%%%%%%%%%%%%%%%%%
%Introduction

\section*{引言}
\addcontentsline{toc}{section}{引言}
\xiaosi

%%%%%%%%%%%%%%%%%%%%%%%%%%%%%%%%%%%%%%%%%%%%%%%%%%%%%%%%%%%%%


%%%%%%%%%%%%%%%%%%%%%%%%%%%%%%%%%%%%%%%%%%%%%%%%%%%%%%%%%%%%%

%Reagents and Devices
\section{试剂与仪器}

\subsection{试剂}


\subsection{仪器}



%%%%%%%%%%%%%%%%%%%%%%%%%%%%%%%%%%%%%%%%%%%%%%%%%%%%%%%%%%%%

%%%%%%%%%%%%%%%%%%%%%%%%%%%%%%%%%%%%%%%%%%%%%%%%%%%%%%%%%%%%
%Methods and Procedures
\section{实验步骤}



\subsection{}

%%%%%%%%%%%%%%%%%%%%%%%%%%%%%%%%%%%%%%%%%%%%%%%%%%%%%%%%%%%
%Results and Analysis
\section{实验结果和分析}

%%%%%%%%%%%%%%%%%%%%%%%%%%%%%%%%%%%%%%%%%%%%%%%%%%%%%%%%%%%%
%References
%Not for now.
\bibliography{References}
\addcontentsline{toc}{section}{}



%%%%%%%%%%%%%%%%%%%%%%%%%%%%%%%%%%%%%%%%%%%%%%%%%%%%%%%%%%%


%%%%%%%%%%%%%%%%%%%%%%%%%%%%%%%%%%%%%%%%%%%%%%%%%%%%%%%%%%%
%Review
\newpage
\section*{实验感想}
\wuhao
\addcontentsline{toc}{section}{实验感想}


\begin{center}
\xiaosan\textbf{}

\xiaosi{李睿(3160102098)}
\end{center}
\wuhao




\end{document}
